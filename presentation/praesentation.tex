%--Pr�ambel
\documentclass{beamer}

\usepackage{ngerman}
\usepackage{graphicx}
\usepackage{pst-circ}
\usepackage{pstricks}
\usepackage{pstricks-add}

\usepackage{pst-node}
\usepackage{pst-plot}
\usepackage{multido}
\usepackage[latin1]{inputenc}
%\usepackage[left=2cm,right=2cm,top=2cm,bottom=2cm]{geometry}
\usepackage{wrapfig}

%Verwendetes Template
\usetheme{Madrid}

%Transparenz nicht aufgedeckter bzw. wieder verdeckter Elemente festlegen
\setbeamercovered{
still covered={\opaqueness<1->{10}},
again covered={\opaqueness<1->{50}}
}


%--Daten der Titelseite
\title[Artefaktkorrektur]{EMG und EOG Artefaktkorrektur \\ in der \\ BCI-Forschung}
\author{Ebner Thomas, Hoheneder Raphael}
\date{\today}

%--Hauptinhalte
\begin{document}

\psset{xunit=0.8cm, yunit=0.8cm}
\psset{tensioncolor=blue, tensionlabelcolor=blue}
\psset{intensitycolor=red, intensitylabelcolor=red}

%%% Titelseite anzeigen
\frame{\titlepage}

% Inhalts�bersicht
\frame{\frametitle{Inhalts�bersicht} 
\tableofcontents
}%endframe


\section{�bersicht}
\frame{\frametitle{�bersicht}

\begin{block}{Was sind Artefakte?}
�berlagerte St�reinfl�sse.
\end{block}

\vspace{5mm}

\begin{block}{EOG}
St�rungen durch Bewegung der Augen
\end{block}

\vspace{5mm}

\begin{block}{EMG}
St�rungen durch Muskelbewegungen.
\end{block}

}%endframe


\frame{\frametitle{�bersicht}

\begin{block}{Warum?}
BCI-Steuerung durch EEG, nicht durch EMG/EOG.
\end{block}

\vspace{5mm}

\begin{block}{Wie?}
Entfernung vs. Erkennung
\end{block}

}%endframe


\section{Methoden}
\frame{\frametitle{Inverse Filterung - Autoregressives Modell}

\begin{block}{Autoregressives Modell}
\begin{equation}
y_t = \sum_i a_i \cdot y_{t-i} + v_t
\end{equation}
\end{block}

\vspace{5mm}

\begin{block}{Inverse}
Inverse des AR-Modells wird auf das EEG Signal angewandt.
\end{block}

}%endframe

\frame{\frametitle{Morphological Component Analysis}

\begin{block}{Zerlegung des Signals}
\end{block}



\begin{block}{Basen}
\begin{itemize}
 \item DCT
 \item Wavelet
 \item Dirac
 \item ...
\end{itemize}

\end{block}

}%endframe



% \frame{\frametitle{Mustererkennung}
% 
% 
% \begin{columns}
% 
% \begin{column}{5cm}
% \begin{block}{Fixpunkte}
% Blickwinkel: gewisse Zeit innerhalb eines gewissen Bereichs
% \end{block}
% 
% \begin{block}{Zwinkern}
%  Erkennen des Zwinkerns, \\
%  Unterscheidung zwischen verschiedenen Typen
% \end{block}
% 
% \end{column}
% 
% 
% \begin{column}{6.5cm}
% \begin{figure}[ht]
%  \centering
%  \includegraphics[width=6.5cm]{./pics/blink.png}
%  % eye.png: 617x341 pixel, 72dpi, 21.76x12.03 cm, bb=0 0 617 341
%  \caption{Signalverlauf beim Zwinkern}
% \end{figure}
% \end{column}
% 
% \end{columns}
% }

%%%%%%%%%%%%%%%%%%%%%%%%%%%%%%%%%%%%%%%%%%%%%%%%%%%%%%%%%%%%%%%%%%%%%%%%%%%%%%%%%%%%%%%%%%%%%%%%%%%%
%%%%%%%%%%%%%%%%%%%%%%%%%%%%%%%%%%%%%%%%%%%%%%%%%%%%%%%%%%%%%%%%%%%%%%%%%%%%%%%%%%%%%%%%%%%%%%%%%%%%
%                                         RAPHIS Part:



\frame{\frametitle{Ende}

\begin{center}
\begin{huge}
 Danke f�r die \\
 \vspace{1cm}
 Aufmerksamkeit!
\end{huge}
\end{center}
}





\end{document}