%% This is file `elsarticle-template-1-num.tex',
%%
%% Copyright 2009 Elsevier Ltd
%%
%% This file is part of the 'Elsarticle Bundle'.
%% ---------------------------------------------
%%
%% It may be distributed under the conditions of the LaTeX Project Public
%% License, either version 1.2 of this license or (at your option) any
%% later version.  The latest version of this license is in
%%    http://www.latex-project.org/lppl.txt
%% and version 1.2 or later is part of all distributions of LaTeX
%% version 1999/12/01 or later.
%%
%% The list of all files belonging to the 'Elsarticle Bundle' is
%% given in the file `manifest.txt'.
%%
%% Template article for Elsevier's document class `elsarticle'
%% with numbered style bibliographic references
%%
%% $Id: elsarticle-template-1-num.tex 149 2009-10-08 05:01:15Z rishi $
%% $URL: http://lenova.river-valley.com/svn/elsbst/trunk/elsarticle-template-1-num.tex $
%%
\documentclass[preprint,12pt]{elsarticle}

%% Use the option review to obtain double line spacing
%% \documentclass[preprint,review,12pt]{elsarticle}

%% Use the options 1p,twocolumn; 3p; 3p,twocolumn; 5p; or 5p,twocolumn
%% for a journal layout:
%% \documentclass[final,1p,times]{elsarticle}
%% \documentclass[final,1p,times,twocolumn]{elsarticle}
%% \documentclass[final,3p,times]{elsarticle}
%% \documentclass[final,3p,times,twocolumn]{elsarticle}
%% \documentclass[final,5p,times]{elsarticle}
%% \documentclass[final,5p,times,twocolumn]{elsarticle}

%% if you use PostScript figures in your article
%% use the graphics package for simple commands
%% \usepackage{graphics}
%% or use the graphicx package for more complicated commands
%% \usepackage{graphicx}
%% or use the epsfig package if you prefer to use the old commands
%% \usepackage{epsfig}

%% The amssymb package provides various useful mathematical symbols
\usepackage{amssymb}
%% The amsthm package provides extended theorem environments
%% \usepackage{amsthm}

%% The lineno packages adds line numbers. Start line numbering with
%% \begin{linenumbers}, end it with \end{linenumbers}. Or switch it on
%% for the whole article with \linenumbers after \end{frontmatter}.
%% \usepackage{lineno}

%% natbib.sty is loaded by default. However, natbib options can be
%% provided with \biboptions{...} command. Following options are
%% valid:

%%   round  -  round parentheses are used (default)
%%   square -  square brackets are used   [option]
%%   curly  -  curly braces are used      {option}
%%   angle  -  angle brackets are used    <option>
%%   semicolon  -  multiple citations separated by semi-colon
%%   colon  - same as semicolon, an earlier confusion
%%   comma  -  separated by comma
%%   numbers-  selects numerical citations
%%   super  -  numerical citations as superscripts
%%   sort   -  sorts multiple citations according to order in ref. list
%%   sort&compress   -  like sort, but also compresses numerical citations
%%   compress - compresses without sorting
%%
%% \biboptions{comma,round}

% \biboptions{}


\journal{VWA}

\begin{document}

\begin{frontmatter}

%% Title, authors and addresses

%% use the tnoteref command within \title for footnotes;
%% use the tnotetext command for the associated footnote;
%% use the fnref command within \author or \address for footnotes;
%% use the fntext command for the associated footnote;
%% use the corref command within \author for corresponding author footnotes;
%% use the cortext command for the associated footnote;
%% use the ead command for the email address,
%% and the form \ead[url] for the home page:
%%
%% \title{Title\tnoteref{label1}}
%% \tnotetext[label1]{}
%% \author{Name\corref{cor1}\fnref{label2}}
%% \ead{email address}
%% \ead[url]{home page}
%% \fntext[label2]{}
%% \cortext[cor1]{}
%% \address{Address\fnref{label3}}
%% \fntext[label3]{}

\title{EMG und EOG Artefaktkorrektur in der BCI-Forschung}

%% use optional labels to link authors explicitly to addresses:
%% \author[label1,label2]{<author name>}
%% \address[label1]{<address>}
%% \address[label2]{<address>}

\author{Ebner Thomas}
\author{Hoheneder Raphael}
\address{Graz University of Technology, Austria}

\begin{abstract}
%% Text of abstract

\end{abstract}

\begin{keyword}
%% keywords here, in the form: keyword \sep keyword

%% MSC codes here, in the form: \MSC code \sep code
%% or \MSC[2008] code \sep code (2000 is the default)

\end{keyword}

\end{frontmatter}

%%
%% Start line numbering here if you want
%%
% \linenumbers

%% main text
\section{Introduction}
\label{Introduction}


Um EEG-Signalverläufe sinnvoll für ein BCI verwenden zu können, ist eine Vorverarbeitung notwendig, da 
den Signalverläufen Störeinflüsse überlagert sind.
Solche Störeinflüsse sind unter anderem EOG- bzw. EMG-Artefakte. 
Unter EOG Artefakten versteht man die Störungen, welche durch Augenbewegungen hervorgerufen werden.
Sie entstehen, durch eine permanente Potentialdifferenz von ca. 10-30mV zwischen Vorder- und Hinterseite
des Auges. \cite{Chen2004} Das Bewegen der Augen führt somit zu Störungen, welche dem EEG Signal überlagert sind.
EMG schließt sämtliche andere Muskelaktivitäten mit ein. Ein Beispiele für solche Artefakte sind
Kauen, Schlucken, Athemzüge, den Kopf oder die Stirn bewegen.
In erster Linie wird man einmal versuchen die Bereiche, in denen Störeinflüsse überlagert sind, ausfindig zu machen.
Eine simple Lösung wäre es, diese zeitlichen Bereiche schlicht und einfach zu ignorieren. Dadurch verliert man
sehr viele wertvolle Information. Aus diesem Grund ist man natürlich bemüht die Überlagerten Störungen herauszufiltern.
Ein vollständiges Herausfiltern der Artefakte ist nicht immer möglich,
wobei es manchmal auch schon Ausreichend ist Artefakte zu erkennen und dementsprechend zu Kennzeichnen.
Speziell bei den ersten Trainingsphasen einer BCI treten besonders viele EMG Artefakte auf. Es muss also sehr darauf
geachtet werden, dass man als Informationsquelle die wirklichen EEG Signale und nicht irgendwelche Artefakte verwendet.
~\cite{Pfurtscheller2007}

\section{Methoden}

\subsection{Inverse Filterung - Autoregressives Modell}

Mithilfe der inversen Filterung kann man EMG Artefakte relativ gut erkennen. Dabei werden zuerst die EEG-Signale
mithilfe eines autoregressives Modells modelliert. Bei einem autoregressiven Modell wird ein Gauss-Verteiltes
Rauschen durch eine Filterstruktur geschickt. Die Filterparameter können beispielsweise mithilfe der Burg Methode
bestimmt werden. Beim Bestimmen der Filterkoeffizienten ist darauf zu achten, dass ein Artefaktfreies EEG-Signal
verwendet wird.

Das Filter hat dabei folgende mathematische Struktur, wobei $v_t$ das Gauss-Verteilte Rauschen, $y_t$ der
Filterausgang zum Zeitpunkt $t$ und $a_i$ die Filterkoeffizienten sind:

\begin{equation}
y_t = \sum_i a_i \cdot y_{t-i} + v_t
\end{equation}

Möchte man nun herausfinden, ob ein EEG-Signalverlauf mit EMG Artefakten kontaminiert ist, verwendet man die Inverse
des vorhin beschriebenen Filters und wendet sie auf den Signalverlauf an. Dieses Ergebnis liefert nun einen Rest.
Überschreitet der RMS dieses Signals einen bestimmten Schwellenwert, so kann man einen bestimmten Bereich als Artefakt markieren.
~\cite{Pfurtscheller2007}





%% The Appendices part is started with the command \appendix;
%% appendix sections are then done as normal sections
%% \appendix

%% \section{}
%% \label{}

%% References
%%
%% Following citation commands can be used in the body text:
%% Usage of \cite is as follows:
%%   \cite{key}          ==>>  [#]
%%   \cite[chap. 2]{key} ==>>  [#, chap. 2]
%%   \citet{key}         ==>>  Author [#]

%% References with bibTeX database:

\bibliographystyle{model1-num-names}
\bibliography{paper.bib}

%% Authors are advised to submit their bibtex database files. They are
%% requested to list a bibtex style file in the manuscript if they do
%% not want to use model1-num-names.bst.

%% References without bibTeX database:

% \begin{thebibliography}{00}

%% \bibitem must have the following form:
%%   \bibitem{key}...
%%

% \bibitem{}

% \end{thebibliography}


\end{document}

%%
%% End of file `elsarticle-template-1-num.tex'.
